%% This is file `elsarticle-template-2-harv.tex',
%%
%% Copyright 2009 Elsevier Ltd
%%
%% This file is part of the 'Elsarticle Bundle'.
%% ---------------------------------------------
%%
%% It may be distributed under the conditions of the LaTeX Project Public
%% License, either version 1.2 of this license or (at your option) any
%% later version.  The latest version of this license is in
%%    http://www.latex-project.org/lppl.txt
%% and version 1.2 or later is part of all distributions of LaTeX
%% version 1999/12/01 or later.
%%
%% The list of all files belonging to the 'Elsarticle Bundle' is
%% given in the file `manifest.txt'.
%%
%% Template article for Elsevier's document class `elsarticle'
%% with harvard style bibliographic references
%%
%% $Id: elsarticle-template-2-harv.tex 155 2009-10-08 05:35:05Z rishi $
%% $URL: http://lenova.river-valley.com/svn/elsbst/trunk/elsarticle-template-2-harv.tex $
%%
\documentclass[preprint,authoryear,12pt]{elsarticle}

%% Use the option review to obtain double line spacing
%%\documentclass[authoryear,preprint,review,12pt]{elsarticle}

%% Use the options 1p,twocolumn; 3p; 3p,twocolumn; 5p; or 5p,twocolumn
%% for a journal layout:
%% \documentclass[final,authoryear,1p,times]{elsarticle}
%% \documentclass[final,authoryear,1p,times,twocolumn]{elsarticle}
%% \documentclass[final,authoryear,3p,times]{elsarticle}
%% \documentclass[final,authoryear,3p,times,twocolumn]{elsarticle}
%% \documentclass[final,authoryear,5p,times]{elsarticle}
%% \documentclass[final,authoryear,5p,times,twocolumn]{elsarticle}

%% if you use PostScript figures in your article
%% use the graphics package for simple commands
%%\usepackage{graphics}
%% or use the graphicx package for more complicated commands
 \usepackage{graphicx}
\usepackage{subcaption}
\graphicspath{ {../img/}{../img/au/} {../img/us/}{../img/in/}{../img/stringency_index/}}
%% or use the epsfig package if you prefer to use the old commands
%% \usepackage{epsfig}
\usepackage{amsmath}
%% The amssymb package provides various useful mathematical symbols
\usepackage{amssymb}
%% The amsthm package provides extended theorem environments
\usepackage{amsthm}
\DeclareMathOperator*{\argmin}{arg\,min}
%% The lineno packages adds line numbers. Start line numbering with
%% \begin{linenumbers}, end it with \end{linenumbers}. Or switch it on
%% for the whole article with \linenumbers after \end{frontmatter}.
%% \usepackage{lineno}

%% natbib.sty is loaded by default. However, natbib options can be
%% provided with \biboptions{...} command. Following options are
%% valid:

%%   round  -  round parentheses are used (default)
%%   square -  square brackets are used   [option]
%%   curly  -  curly braces are used      {option}
%%   angle  -  angle brackets are used    <option>
%%   semicolon  -  multiple citations separated by semi-colon (default)
%%   colon  - same as semicolon, an earlier confusion
%%   comma  -  separated by comma
%%   authoryear - selects author-year citations (default)
%%   numbers-  selects numerical citations
%%   super  -  numerical citations as superscripts
%%   sort   -  sorts multiple citations according to order in ref. list
%%   sort&compress   -  like sort, but also compresses numerical citations
%%   compress - compresses without sorting
%%   longnamesfirst  -  makes first citation full author list
%%
\biboptions{square,comma,numbers}

% \biboptions{}

\journal{ }

\begin{document}
	
	\begin{frontmatter}
		
		%% Title, authors and addresses
		
		%% use the tnoteref command within \title for footnotes;
		%% use the tnotetext command for the associated footnote;
		%% use the fnref command within \author or \address for footnotes;
		%% use the fntext command for the associated footnote;
		%% use the corref command within \author for corresponding author footnotes;
		%% use the cortext command for the associated footnote;
		%% use the ead command for the email address,
		%% and the form \ead[url] for the home page:
		%%
		%% \title{Title\tnoteref{label1}}
		%% \tnotetext[label1]{}
		%% \author{Name\corref{cor1}\fnref{label2}}
		%% \ead{email address}
		%% \ead[url]{home page}
		%% \fntext[label2]{}
		%% \cortext[cor1]{}
		%% \address{Address\fnref{label3}}
		%% \fntext[label3]{}
		
		\title{A Study on The Effectiveness of Lock-down Measures to Control The Spread of COVID-19}
		
		%% use optional labels to link authors explicitly to addresses:
		%% \author[label1,label2]{<author name>}
		%% \address[label1]{<address>}
		%% \address[label2]{<address>}
		
		\author[a1]{Subhas Kumar Ghosh\corref{cor1}}
		\ead{subhas.ghosh@cba.com.au}
		\author[a2]{Sai Shanmukha Narumanchi}
		\ead{sai@cs.siu.edu}
		\author[a3]{Sachchit Ghosh}
		\ead{sgho2841@uni.sydney.edu.au}
		
		\cortext[cor1]{Corresponding author}
		\address[a1]{Commonwealth Bank of Australia, Sydney, New South Wales, 2000, Australia}
		\address[a2]{Department of Computer Science, Southern Illinois University, Carbondale, IL 62901, USA.}
		\address[a3]{The University of Sydney, Camperdown, NSW 2006, Australia}
		
		\begin{abstract}
			%% Text of abstract
			
		\end{abstract}
		
		\begin{keyword}
			%% keywords here, in the form: keyword \sep keyword
			COVID-19 \sep Lock down \sep Mathematical modeling \sep Epidemic \sep Economic Impact
			
			%% MSC codes here, in the form: \MSC code \sep code
			%% or \MSC[2008] code \sep code (2000 is the default)
			
		\end{keyword}
		
	\end{frontmatter}
	
	% \linenumbers
	
	%% main text
\section{Introduction}
	\label{SEC1}
	In December 2019, an outbreak occurred in Wuhan, China involving a zoonotic coronavirus, similar to SARS coronavirus and MERS coronavirus  \cite{taaa021}. The virus has been named Severe Acute Respiratory Syndrome Coronavirus 2 (SARS-CoV-2), and the disease caused by the virus has been named the coronavirus disease 2019 (COVID-19). Since then the ongoing pandemic has infected more than 9 million people and has caused more than 467 thousand deaths worldwide.
	
	Since the initial outbreak, several different studies have tried to estimate the number of infections \cite{GN2020} that stem from a single infected patient in order to predict the potential for transmission of the COVID-19 virus. In most cases, it was seen that $R_0 > 1$, implying exponential growth through infection of a vulnerable population. Original estimates placed mortality rates for individuals at high risk at  4.46 \% with those suffering from cardiovascular or kidney disease having even greater susceptibility \cite{BPH2020}. 
	
	The SARS-CoV-2 virus has no available treatment as the pathways for proliferation and pathogenesis are still unclear \cite{RIS2020}.  Current treatments are based on those effective on strains of the previous SARS coronavirus and MERS coronavirus. The SARS-CoV-2  virus is able to replicate rapidly during the asymptomatic phase and affect the lungs and respiratory tract, resulting in pneumonia, hypoxia, and acute respiratory distress \cite{PSL2020}. Infected patients are directly dependant on external ventilation in most cases. 
	
	
	With the increasing pressure on the health systems due to reliance on intensive care units or non-invasive ventilation, health strategies like social distancing were implemented. The concern was to ensure the number of infected patients does not exceed the health system’s ability to cope with it. It also focused on increasing the capacities of available health systems at the time.
	
	
	Under the conditions at the time, with a highly pathogenic SARS-CoV-2 that is able to spread asymptomatically during its incubation stage through a vulnerable population, policymakers were to contain the spread of the infection, minimize stress on the health systems and ensure public safety. This was done by issuing orders for widespread lockdown and implementing social distancing measures. All non-essential businesses and services were shut down until further notice. 
	

	Taking measures to reduce stress on the health sector and diminishing the number of infected patients is important to end the pandemic, and understanding the effectiveness of a lockdown enables the distinction of good safety measures from bad ones. Analyzing a synthetic control allows us to understand whether the decisions made were optimal and resulted in a reduction of burden on the healthcare system, and broke chain of transmission, preventing its spread and reducing the reproductive rate of the virus.
	
	How can we measure these benefits and compare with economic losses
	
	\subsection{Our contribution}
	Defining how to measure these benefits.
	
	Tools for measuring them.
	
	
	Difficulty in assessing: different level of compliance, different cultural practices - hidden variables
	
	\subsection{Related Works}
	Other modeling approaches. SIR-F, DCM, Agent, Hybrid - not post fact A/B testing tools.
	
	\subsection{Tools}
	As stated above, our objective is to study the effects of government response at an aggregate level in terms of life saved, and limiting the number of cases that requires hospitalization. Such interventions can effectively be studied at a comparative level. In other words, if we have data for the evolution of aggregate outcomes, e.g. number of confirmed cases and deaths, when policy is applied in a group under study versus when the same policy is not applied in a control group. However, government policies were applied at different level across a geographic region.  We do not have a mechanism to conduct a randomized trial. Hence, we consider constructing synthetic control method\cite{ap08746, JMLR18, AMSS19}. In a synthetic control set up, where observational data is available for different groups, we can construct a synthetic or virtual control group by combining measurements from alternatives (or donors). In following we provide a brief overview of Multi-dimensional Robust Synthetic Control following\cite{AMSS19}.\par
	
	Suppose that observations from $N$ different geographically distinct groups or units are indexed by $i \in [N]$ in $T$ time periods (days) indexed by $j \in [T]$. Let $k \in [K]$ be the metrics of interest (e.g. number of confirmed cases, number of deaths, number of tests conducted, etc.). By $M_{ijk}$ we denote the ground-truth measurement of interest, and by $X_{ijk}$, an observation of this measurement with some noise. Let $1 \leq T_0 \leq T$ be the time instance in which our group of interest experiences an intervention, namely a government response to control the spread (e.g. stay home order, school or business closure, or mass vaccination). Without loss of generality we consider unit $i = 1$ (say, New York) and metric $k = 1$ (say, number of deaths) as our unit and metric of interest respectively.\par
	
	Our objective now is to estimate the trajectory of metric of interest $k = 1$  for unit $i = 1$ if no government response to control the spread had occurred. In order to do that we will use the trajectory associated with the donor units ($2 \leq i \leq N$ ), and metrics $k \in [K ]$. In following we make two assumptions: (1) for all $2 \leq i \leq N$, $k \in [K]$ and $j \in [T]$, we have  $X_{ijk} = M_{ijk} + \epsilon_{ijk}$ where, $\epsilon_{ijk}$ is the observational noise, and (2) Same model is obeyed by $i=1$ in pre--intervention period, i.e. for all $j \in [T_0]$ and $k \in [K]$ we have $X_{1jk} = M_{1jk} + \epsilon_{1jk}$. As described by authors in \cite{AMSS19}, in following we also assume that for unit $i=1$, we only observe the measurement $X_{1jk}$ for pre-intervention period, i.e. for all $j \in [T_0]$ and $k \in [K]$. Our objective is to compute a counterfactual sequence of observation $M_{1jk}$ for the time period $j \in [T]$, and $k \in [K]$, and in specific for $T_0 \leq j \leq T$, and $k = 1$, using synthetic version of unit $i=1$.\par
	
	Define $\mathcal{M} = [M_{ijk}] \in \mathbb{R}^{N \times T \times K}$. $\mathcal{M}$ is assumed to have a few well behaved properties as required by the algorithm, namely, it must be approximately low-rank and boundedness of $\left|M_{ijk}\right|$ (for details see \cite{AMSS19}). To check whether our model assumption holds in practice, we consider $N=185, T=150, K=2$, with $185$ countries as units. We consider number of confirmed cases and number of deceased as two metrics over $150$ days between January 22, 2020 and June 20, 2020. For assumption to hold, data matrix corresponding to number of confirmed cases and number of deceased and their combination should be approximated by a low-rank matrix. 
	
	\begin{figure}[ht]
		\includegraphics[width=\textwidth]{rd}
		\caption{The Singular Value spectrum for all countries (of dimensions $185 \times 150$, showing top 20 singular values, in descending order.} 
		\label{fig1} 
	\end{figure}
	
	As shown in Figure \ref{fig1}, the spectrum of the top 20 singular values (sorted in descending order) for each matrix. The plots clearly support the implications that most of the spectrum is concentrated within the top 5 principal components. Same conclusion holds true when units are states of United States, and when we consider only countries in European Union.\par
	
	Let $\mathcal{Z} \in \mathbb{R}^{(N-1) \times T \times K}$ corresponding to donor units, and $X_1 \in \mathbb{R}^{1 \times T_0 \times K}$ correspond to unit under intervention. We obtain $\hat{\mathcal{M}}$ from $\mathcal{Z}$ after applying a hard singular value thresholding. Subsequently, weights are learned using linear regression by computing
	
	\begin{equation*}
	\hat{\beta} = \argmin_{v \in \mathbb{R}^{(N-1)} } \left\| X_1 - v^T \hat{\mathcal{M}}_{T_0}\right\|^2_2
	\end{equation*}
	
	For every $k \in [K]$, the corresponding estimated counterfactual means for the treatment unit is then defined as
	
	\begin{equation*}
	\hat{\mathcal{M}}_1^{(k)} = \hat{\beta}^T \hat{\mathcal{M}}^{(k)}
	\end{equation*}
	
\section{Methods}
	\label{SEC2}
	\subsection{Overview}
	As described in Section \ref{SEC1}, we use Multi-dimensional Robust Synthetic Control to construct a synthetic control for the treatment unit using data from multiple control units or donor group using pre-intervention period data.  The synthetic control is then used for estimating the counterfactual in the post-intervention period. In our setup intervention date is typically the date when a stay-home order or lock-down was declared for the treatment unit.  However, government policy may have been applied over time with different level of stringency measures. 
	
	To understand this we use stringency and policy indices data from OxCGRT \cite{HWP2020}, which records the strictness of policies that restrict people’s behavior and includes 8 different measures - e.g. school  and workplace closure, cancellation of public events, restrictions on gathering size etc. Figure \ref{fig2}, shows the plot of stringency index, with important events timeline. 
	
	\begin{figure}
		\begin{subfigure}[b]{0.9\textwidth}
			\includegraphics[width=1\linewidth]{GRSI_ITALY}
		\end{subfigure}
		
		\begin{subfigure}[b]{0.9\textwidth}
			\includegraphics[width=1\linewidth]{GRSI_INDIA}
		\end{subfigure}
		
			\begin{subfigure}[b]{0.9\textwidth}
			\includegraphics[width=1\linewidth]{GRSI_US}
		\end{subfigure}
		\label{fig2} 
		\caption[Stringency index, with important events timeline]{(a) Italy (b) India (c) United States.}
	\end{figure}

	\subsection{Data Source}
	
	\subsection{Setup}


\section{Results}
\label{SEC3}
	
	India - 4 stages of lock-down - effect of each stages - results on prediction by Synthetic control


	Singapore - recurrence - change in projection on those dates
	
	US - compare prediction models data vs. Synthetic control projection vs actual by state by start and end of lock-down dates (what are the control group in each cases)
	
	AU NZ South K
	
	Sweden
	
	Measured results. 
	
	\section{Discussion}
	\label{SEC4}
	
	
	\section{Concluding Remarks}
	\label{SEC5}
	%% References
	
	\bibliographystyle{elsarticle-num}
	\bibliography{lockdown.bib}	
\end{document}

%%
%% End of file `elsarticle-template-2-harv.tex'.
