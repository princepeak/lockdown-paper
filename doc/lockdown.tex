%% This is file `elsarticle-template-2-harv.tex',
%%
%% Copyright 2009 Elsevier Ltd
%%
%% This file is part of the 'Elsarticle Bundle'.
%% ---------------------------------------------
%%
%% It may be distributed under the conditions of the LaTeX Project Public
%% License, either version 1.2 of this license or (at your option) any
%% later version.  The latest version of this license is in
%%    http://www.latex-project.org/lppl.txt
%% and version 1.2 or later is part of all distributions of LaTeX
%% version 1999/12/01 or later.
%%
%% The list of all files belonging to the 'Elsarticle Bundle' is
%% given in the file `manifest.txt'.
%%
%% Template article for Elsevier's document class `elsarticle'
%% with harvard style bibliographic references
%%
%% $Id: elsarticle-template-2-harv.tex 155 2009-10-08 05:35:05Z rishi $
%% $URL: http://lenova.river-valley.com/svn/elsbst/trunk/elsarticle-template-2-harv.tex $
%%
\documentclass[preprint,authoryear,12pt]{elsarticle}

%% Use the option review to obtain double line spacing
%%\documentclass[authoryear,preprint,review,12pt]{elsarticle}

%% Use the options 1p,twocolumn; 3p; 3p,twocolumn; 5p; or 5p,twocolumn
%% for a journal layout:
%% \documentclass[final,authoryear,1p,times]{elsarticle}
%% \documentclass[final,authoryear,1p,times,twocolumn]{elsarticle}
%% \documentclass[final,authoryear,3p,times]{elsarticle}
%% \documentclass[final,authoryear,3p,times,twocolumn]{elsarticle}
%% \documentclass[final,authoryear,5p,times]{elsarticle}
%% \documentclass[final,authoryear,5p,times,twocolumn]{elsarticle}

%% if you use PostScript figures in your article
%% use the graphics package for simple commands
%%\usepackage{graphics}
%% or use the graphicx package for more complicated commands
 \usepackage{graphicx}
 \graphicspath{ {../img/au/} {../img/us/}{../img/in/}}
%% or use the epsfig package if you prefer to use the old commands
%% \usepackage{epsfig}

%% The amssymb package provides various useful mathematical symbols
\usepackage{amssymb}
%% The amsthm package provides extended theorem environments
\usepackage{amsthm}

%% The lineno packages adds line numbers. Start line numbering with
%% \begin{linenumbers}, end it with \end{linenumbers}. Or switch it on
%% for the whole article with \linenumbers after \end{frontmatter}.
%% \usepackage{lineno}

%% natbib.sty is loaded by default. However, natbib options can be
%% provided with \biboptions{...} command. Following options are
%% valid:

%%   round  -  round parentheses are used (default)
%%   square -  square brackets are used   [option]
%%   curly  -  curly braces are used      {option}
%%   angle  -  angle brackets are used    <option>
%%   semicolon  -  multiple citations separated by semi-colon (default)
%%   colon  - same as semicolon, an earlier confusion
%%   comma  -  separated by comma
%%   authoryear - selects author-year citations (default)
%%   numbers-  selects numerical citations
%%   super  -  numerical citations as superscripts
%%   sort   -  sorts multiple citations according to order in ref. list
%%   sort&compress   -  like sort, but also compresses numerical citations
%%   compress - compresses without sorting
%%   longnamesfirst  -  makes first citation full author list
%%
\biboptions{square,comma}

% \biboptions{}

\journal{ }

\begin{document}
	
	\begin{frontmatter}
		
		%% Title, authors and addresses
		
		%% use the tnoteref command within \title for footnotes;
		%% use the tnotetext command for the associated footnote;
		%% use the fnref command within \author or \address for footnotes;
		%% use the fntext command for the associated footnote;
		%% use the corref command within \author for corresponding author footnotes;
		%% use the cortext command for the associated footnote;
		%% use the ead command for the email address,
		%% and the form \ead[url] for the home page:
		%%
		%% \title{Title\tnoteref{label1}}
		%% \tnotetext[label1]{}
		%% \author{Name\corref{cor1}\fnref{label2}}
		%% \ead{email address}
		%% \ead[url]{home page}
		%% \fntext[label2]{}
		%% \cortext[cor1]{}
		%% \address{Address\fnref{label3}}
		%% \fntext[label3]{}
		
		\title{A Study on The Effectiveness of Lock-down Measures to Control The Spread of COVID-19}
		
		%% use optional labels to link authors explicitly to addresses:
		%% \author[label1,label2]{<author name>}
		%% \address[label1]{<address>}
		%% \address[label2]{<address>}
		
		\author[a1]{Subhas Kumar Ghosh\corref{cor1}}
		\ead{subhas.ghosh@cba.com.au}
		\author[a2]{Sai Shanmukha Narumanchi}
		\ead{sai@cs.siu.edu}
		\author[a3]{Sachchit Ghosh}
		\ead{sgho2841@uni.sydney.edu.au}
		
		\cortext[cor1]{Corresponding author}
		\address[a1]{Commonwealth Bank of Australia, Sydney, New South Wales, 2000, Australia}
		\address[a2]{Department of Computer Science, Southern Illinois University, Carbondale, IL 62901, USA.}
		\address[a3]{The University of Sydney, Camperdown, NSW 2006, Australia}
		
		\begin{abstract}
			%% Text of abstract
			
		\end{abstract}
		
		\begin{keyword}
			%% keywords here, in the form: keyword \sep keyword
			COVID-19 \sep Lock down \sep Mathematical modeling \sep Epidemic \sep Economic Impact
			
			%% MSC codes here, in the form: \MSC code \sep code
			%% or \MSC[2008] code \sep code (2000 is the default)
			
		\end{keyword}
		
	\end{frontmatter}
	
	% \linenumbers
	
	%% main text
	\section{Introduction}
	\label{SEC1}
	In December 2019, an outbreak occurred in Wuhan, China involving a zoonotic coronavirus, similar to SARS coronavirus and MERS coronavirus  \cite{taaa021}. The virus has been named Severe Acute Respiratory Syndrome Coronavirus 2 (SARS-CoV-2), and the disease caused by the virus has been named the coronavirus disease 2019 (COVID-19). Since then the ongoing pandemic has infected more than 9 million people and has caused more than 467 thousand deaths worldwide.
	
	Since the initial outbreak, several different studies have tried to estimate the number of infections \cite{GN2020} that stem from a single infected patient in order to predict the potential for transmission of the COVID-19 virus. In most cases, it was seen that $R_0 > 1$, implying exponential growth through infection of a vulnerable population. Original estimates placed mortality rates for individuals at high risk at  4.46 \% with those suffering from cardiovascular or kidney disease having even greater susceptibility \cite{BPH2020}. 
	
	The SARS-CoV-2 virus has no available treatment as the pathways for proliferation and pathogenesis are still unclear \cite{RIS2020}.  Current treatments are based on those effective on strains of the previous SARS coronavirus and MERS coronavirus. The SARS-CoV-2  virus is able to replicate rapidly during the asymptomatic phase and affect the lungs and respiratory tract, resulting in pneumonia, hypoxia, and acute respiratory distress \cite{PSL2020}. To reduce 
	
	Under the conditions at the time, with a highly pathogenic SARS-CoV-2 that is able to spread asymptomatically during its incubation stage through a vulnerable population, policymakers were to contain the spread of the infection, minimize stress on the health systems and ensure public safety. This was done by issuing orders for widespread lockdown and implementing social distancing measures. All non-essential businesses and services were shut down until further notice. 
	
	The impact of this lockdown directly
	
	
	Motivate the requirements for analyzing effectiveness of lock-down.   What were the benefits? Reducing burden on healthcare system, saving loss of life, and reducing $R_0$ by breaking transmission chain. 
	
	How can we measure these benefits and compare with the economic losses?
	
	\subsection{Our contribution}
	Defining how to measure these benefits.
	
	Tools for measuring them.
	
	
	Difficulty in assessing: different level of compliance, different cultural practices - hidden variables
	
	\subsection{Related Works}
	Other modeling approaches. SIR-F, DCM, Agent, Hybrid - not post fact A/B testing tools.
	
	\subsection{Tools}
	
	$m-RSC$ \cite{ap08746, JMLR18}, Trend analysis, 
	
	
	
	\section{The setup}
    \label{SEC2}
	Data - Data Source - Dims [infected, fatal, recovered, No of tests] 
	
	
	
	\section{Results}
	\label{SEC3}
	
	India - 4 stages of lock-down - effect of each stages - results on prediction by Synthetic control


	Singapore - recurrence - change in projection on those dates
	
	US - compare prediction models data vs. Synthetic control projection vs actual by state by start and end of lock-down dates (what are the control group in each cases)
	
	AU NZ South K
	
	Sweden
	
	Measured results. 
	
	\section{Discussion}
	\label{SEC4}
	
	
	\section{Concluding Remarks}
	\label{SEC5}
	%% References
	
	\bibliographystyle{elsarticle-num}
	\bibliography{lockdown.bib}	
\end{document}

%%
%% End of file `elsarticle-template-2-harv.tex'.
